\documentclass{article}
\usepackage{amsmath, amsthm, amssymb}
\usepackage[spanish]{babel}
\title{Condición de Integrabilidad de Riemann}
\author{Alex David Montero Garay y Erick Herández Peón }
\date{\today}

\begin{document}
	\maketitle
	
	\section{Equivalencias de Integrabilidad Riemann}
	Sea \( f: [a, b] \to \mathbb{R} \) una función acotada. Las siguientes proposiciones son equivalentes:
	
	\begin{enumerate}
		\item \( f \) es Riemann integrable en \([a, b]\).
		\item Para todo \( \varepsilon > 0 \), existe una partición \( P_\varepsilon \) tal que para toda partición más fina \( P' \supseteq P_\varepsilon \) y cualquier elección de puntos \( \chi \):
		
		
		\[
		|\sigma(f, P', \chi) - I| < \varepsilon,
		\]
		
		
		donde \( \sigma(f, P', \chi) \) es una suma de Riemann y \( I = \int_a^b f(x)\, dx \).
		\item La integral superior e inferior coinciden:
		
		
		\[
		\overline{I} = \underline{I} 
		\]
		
		
	\end{enumerate}
	
	\subsection*{Demostración de las equivalencias}
	
	\begin{proof}[Demostración (1) \(\Rightarrow\) (2)]
		Supongamos que \( f \) es Riemann integrable. Por el criterio de Darboux, para todo \( \varepsilon > 0 \), existe una partición \( P_\varepsilon \) tal que:
		
		
		\[
		U(P_\varepsilon, f) - L(P_\varepsilon, f) < \varepsilon.
		\]
		
		
		Si \( P' \supseteq P_\varepsilon \), entonces:
		
		
		\[
		L(P_\varepsilon, f) \leq L(P', f) \leq \sigma(f, P', \chi) \leq U(P', f) \leq U(P_\varepsilon, f).
		\]
		
		
		Como \( I = \int_a^b f(x)\, dx \), se cumple:
		
		
		\[
		|\sigma(f, P', \chi) - I| \leq U(P', f) - L(P', f) < \varepsilon. \quad \qedhere
		\]
	\end{proof}
	
	\begin{proof}[Demostración (2) \(\Rightarrow\) (3)]
		Dado \( \varepsilon > 0 \), por hipótesis existe \( P_\varepsilon \) tal que para todo \( P' \supseteq P_\varepsilon \):
		
		
		\[
		I - \varepsilon < \sigma(f, P', \chi) < I + \varepsilon.
		\]
		
		
		Tomando supremo e ínfimo sobre las sumas de Riemann:
		
		
		\[
		I - \varepsilon \leq L(P', f) \leq \underline{(I)}  \leq \overline{(I)}  \leq U(P', f) \leq I + \varepsilon.
		\]
		
		
		Como \( \varepsilon \) es arbitrario, se concluye:
		
		
		\[
		\overline{I} = \underline{I} = I. \quad \qedhere
		\]
	\end{proof}
	
	\begin{proof}[Demostración (3) \(\Rightarrow\) (1)]
		Por definición, \( f \) es Riemann integrable si y solo si:
		
		
		\[
		\overline{I} = \underline{I} .
		\]
		
		
		La igualdad en (3) implica directamente la integrabilidad, con \( I = \int_a^b f(x)\, dx \). \quad \qedhere
	\end{proof}
	
	
	
	\[
	\boxed{
		\begin{aligned}
			&1 \implies 2 \implies 3 \implies 1 \\
			&\text{Equivalencia completa.}
		\end{aligned}
	}
	\]
	
	
\end{document}

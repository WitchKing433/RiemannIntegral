\documentclass{article}
\usepackage{amsmath, amsthm, amssymb}
\usepackage[spanish]{babel}
\title{Condición necesaria y suficiente de Integrabilidad de Riemann}
\author{Alex David Montero Garay y Erick Hernández Peón }
\date{}

\begin{document}
	\maketitle
	
	\section*{Equivalencias de Integrabilidad Riemann}
	Sea \( f: [a, b] \to \mathbb{R} \) una función acotada. 
	Demostrar que:
	\[
	   f \in R:[a,b] \iff \forall \varepsilon >0   [\exists P_\varepsilon :  P' \supseteq P_\varepsilon 	|\sigma(f, P', \chi) - I| < \varepsilon] \iff 	\overline{ I } = \underline{I}.
	\]
 Las siguientes proposiciones son equivalentes:
	\begin{enumerate}
		\item f es Riemann integrable en [a, b]. \[  f \in R:[a,b] \]
		\item  para todo epsilon, existe una partición P subepsilon tal que para toda partición más fina P subepsilon se encuentre contenida en la partición P' y cualquier elección de puntos Xi: 
		\[ \forall \varepsilon>0   [\exists P_\varepsilon :  P' \supseteq P_\varepsilon 	|\sigma(f, P', \chi) - I| < \varepsilon]
		\]
		
	
		\item La integral superior e inferior coinciden:
		
		
		\[
		\overline{I} = \underline{I} 
		\]
		
		
	\end{enumerate}
	
	\section*{Demostración de las equivalencias}
	
	\begin{proof}\textbf{(1) \(\Rightarrow\) (2)}.\\
	Hipótesis:
	\( f \) es Riemann integrable en \([a, b]\), es decir, la integral superior \( \overline{I} \) y la integral inferior \( \underline{I} \) coinciden:  
	\[
	\overline{I} = \underline{I} = I.
	\]
	
	Tesis:  
	Para todo \( \varepsilon > 0 \), existe una partición \( P_\varepsilon \) tal que, para cualquier refinamiento \( P' \supseteq P_\varepsilon \) y cualquier elección de puntos de muestra \( \chi \), se cumple:  
	\[
	|\sigma(f, P', \chi) - I| < \varepsilon.
	\]
	
	I. Integral superior e inferior:
	\[
	\overline{I} = \inf_{P} U(f, P), \quad \underline{I} = \sup_{P} L(f, P),
	\]  
	donde \( U(f, P) \) y \( L(f, P) \) son las sumas de Darboux (superior e inferior) asociadas a una partición \( P \).
	
	II. Criterio de Darboux para integrabilidad: 
	Si \( f \) es Riemann integrable, entonces para todo \( \varepsilon > 0 \), existe una partición \( P_\varepsilon \) tal que:  
	\[
	U(f, P_\varepsilon) - L(f, P_\varepsilon) < \varepsilon.
	\]
	
	Por la integrabilidad de \( f \), aplicamos el criterio de Darboux. Para el \( \varepsilon > 0 \) dado, existe una partición \( P_\varepsilon = \{x_0, x_1, \ldots, x_n\} \) de \([a, b]\) tal que:  
	\[
	U(f, P_\varepsilon) - L(f, P_\varepsilon) < \varepsilon.
	\]
	Sea \( P' \supseteq P_\varepsilon \) un afinamiento de \( P_\varepsilon \).\\ 
	\textbf{Por propiedades de las sumas de Darboux:}  .\\
	a). Monotonía:
	\[
	L(f, P_\varepsilon) \leq L(f, P') \leq U(f, P') \leq U(f, P_\varepsilon).
	\]  
	b). 
	\[
	U(f, P') - L(f, P') \leq U(f, P_\varepsilon) - L(f, P_\varepsilon) < \varepsilon.
	\]
	
	Sea \( \sigma(f, P', \chi) = \sum_{i=1}^m f(\chi_i) \Delta x_i \) una suma de Riemann asociada a \( P' \) y puntos de muestra \( \chi = \{\chi_1, \ldots, \chi_m\} \). Por definición:  
	\[
	L(f, P') \leq \sigma(f, P', \chi) \leq U(f, P').
	\]  
	Además, como \( f \) es integrable, \( I \) satisface:  
	\[
	L(f, P') \leq I \leq U(f, P').
	\]  
	
	Desigualdad triangular para la diferencia:  
	\[
	|\sigma(f, P', \chi) - I| \leq \max\left\{ |U(f, P') - I|, |L(f, P') - I| \right\}.
	\]  
	
	Pero, \( U(f, P') - L(f, P') < \varepsilon \), y como \( I \in [L(f, P'), U(f, P')] \):  
	\[
	|U(f, P') - I| \leq U(f, P') - L(f, P') < \varepsilon,
	\]  
	\[
	|L(f, P') - I| \leq U(f, P') - L(f, P') < \varepsilon.
	\]  
	
	Por lo tanto:  
	\[
	|\sigma(f, P', \chi) - I| < \varepsilon.
	\]
	
	Conclusión:
	Si \( f \) es Riemann integrable en \([a, b]\), entonces para todo \( \varepsilon > 0 \), existe una partición \( P_\varepsilon \) tal que cualquier afinamiento \( P' \supseteq P_\varepsilon \) cumple \( |\sigma(f, P', \chi) - I| < \varepsilon \). Esto demuestra la implicación:  
	\[
	f \in \mathcal{R}[a, b] \implies \forall \varepsilon > 0,\, \exists P_\varepsilon : P' \supseteq P_\varepsilon \implies |\sigma(f, P', \chi) - I| < \varepsilon.
	\]
	
	\[
	\boxed{
		f \in \mathcal{R}[a, b] \implies \forall \varepsilon > 0,\, \exists P_\varepsilon : P' \supseteq P_\varepsilon \implies |\sigma(f, P', \chi) - I| < \varepsilon
	}
	\]
		
	\end{proof}
	
	\begin{proof}\textbf{(2) \(\Rightarrow\) (3)}.\\
	Hipótesis: 
	Para todo \( \varepsilon > 0 \), existe una partición \( P_\varepsilon \) tal que, para cualquier afinamiento \( P' \supseteq P_\varepsilon \) y cualquier elección de puntos de muestra \( \chi \):  
	\[
	|\sigma(f, P', \chi) - I| < \varepsilon.
	\].\\
	Tesis: 
	\[
	\overline{I} = \underline{I}.
	\]
	
	I. Integral superior (Darboux):
	\[
	\overline{I} = \inf_{P} U(f, P),
	\]  
	donde \( U(f, P) = \sum_{i=1}^n M_i \Delta x_i \), con \( M_i = \sup_{[x_{i-1}, x_i]} f(x) \).  
	
	II. Integral inferior (Darboux): 
	\[
	\underline{I} = \sup_{P} L(f, P),
	\]  
	donde \( L(f, P) = \sum_{i=1}^n m_i \Delta x_i \), con \( m_i = \inf_{[x_{i-1}, x_i]} f(x) \).  
	
	III. Propiedad fundamental:  
	Para toda partición \( P \), \( L(f, P) \leq \underline{I} \leq \overline{I} \leq U(f, P) \).  
			
	Tomando supremo e ínfimo sobre las sumas de Riemann:
		
		
		\[
		I - \varepsilon \leq L(P', f) \leq \underline{(I)}  \leq \overline{(I)}  \leq U(P', f) \leq I + \varepsilon.
		\]			
	Como \( \varepsilon \) es arbitrario, se concluye:
		
		
		\[
		\overline{I} = \underline{I} = I. \quad \qedhere
		\]
	Por hipótesis, para todo \( \varepsilon > 0 \), existe \( P_\varepsilon \) tal que:  
		\[
		|\sigma(f, P', \chi) - I| < \varepsilon \quad \forall P' \supseteq P_\varepsilon, \, \forall \chi.
		\]  
	Esto implica:  
		\[
		I - \varepsilon < \sigma(f, P', \chi) < I + \varepsilon.
		\]  
	Como \( U(f, P') = \sup_{\chi} \sigma(f, P', \chi) \), se tiene:  
		\[
		U(f, P') \leq I + \varepsilon.
		\]  
		
	Cota para la suma inferior \( L(f, P') \):
	Como \( L(f, P') = \inf_{\chi} \sigma(f, P', \chi) \), se tiene:  
		\[
		L(f, P') \geq I - \varepsilon.
		\]  
		\textbf{Acotación de \( \overline{I} \):}
		Por definición, \( \overline{I} \) es el ínfimo de todas las sumas superiores. Dado que \( U(f, P') \leq I + \varepsilon \) para todo afinamiento \( P' \supseteq P_\varepsilon \):  
		\[
		\overline{I} \leq I + \varepsilon.
		\]  
		Como \( \varepsilon > 0 \) es arbitrario, tomando \( \varepsilon \to 0^+ \):  
		\[
		\overline{I} \leq I.
		\]  
		\textbf{Acotación de \( \underline{I} \):}
		Por definición, \( \underline{I} \) es el supremo de todas las sumas inferiores. Dado que \( L(f, P') \geq I - \varepsilon \) para todo afinamiento \( P' \supseteq P_\varepsilon \):  
		\[
		\underline{I} \geq I - \varepsilon.
		\]  
		Tomando \( \varepsilon \to 0^+ \):  
		\[
		\underline{I} \geq I.
		\]  
		
		De los resultados anteriores:  
		\[
		\overline{I} \leq I \quad \text{y} \quad \underline{I} \geq I.
		\]  
		Pero por la propiedad fundamental \( \underline{I} \leq \overline{I} \), se concluye:  
		\[
		\underline{I} \geq I \geq \overline{I} \geq \underline{I}.
		\]  
		Esto solo es posible si:  
		\[
		\overline{I} = \underline{I} = I.
		\]
		Conclusión:
		La condición \( \forall \varepsilon > 0,\, \exists P_\varepsilon : P' \supseteq P_\varepsilon \implies |\sigma(f, P', \chi) - I| < \varepsilon \) implica necesariamente que las integrales superior e inferior coinciden (\( \overline{I} = \underline{I} \))
		\[
		\boxed{
			\forall \varepsilon > 0,\, \exists P_\varepsilon : P' \supseteq P_\varepsilon \implies |\sigma(f, P', \chi) - I| < \varepsilon \quad \implies \quad \overline{I} = \underline{I}
		}
		\]
	\end{proof}
	
	\begin{proof}\textbf{(3) \(\Rightarrow\)(1)}:\\
		Hipótesis:
		\[
		\overline{I} = \underline{I} = I,
		\]  
		donde \( \overline{I} \) es la integral superior de Darboux y \( \underline{I} \) la integral inferior.\\
		Tesis: 
		\( f \) es Riemann integrable en \([a, b]\), es decir:  
		\[
		\forall \varepsilon > 0,\, \exists P_\varepsilon : \forall P' \supseteq P_\varepsilon,\, |\sigma(f, P', \chi) - I| < \varepsilon.
		\]
		
		  Integral superior:
		 \[
		 \overline{I} = \inf_{P} U(f, P), \quad U(f, P) = \sum_{i=1}^n M_i \Delta x_i, \quad M_i = \sup_{[x_{i-1}, x_i]} f(x).
		 \]  
		 Integral inferior:
		 \[
		 \underline{I} = \sup_{P} L(f, P), \quad L(f, P) = \sum_{i=1}^n m_i \Delta x_i, \quad m_i = \inf_{[x_{i-1}, x_i]} f(x).
		 \]  
		 
		Criterio de Darboux:  
		\( f \) es Riemann integrable si y solo si:  
		\[
		\forall \varepsilon > 0,\, \exists P_\varepsilon : U(f, P_\varepsilon) - L(f, P_\varepsilon) < \varepsilon.
		\]
		Aplicación del Criterio de Darboux
		Dado que \( \overline{I} = \underline{I} = I \), por definición de ínfimo y supremo:  
		- Para todo \( \varepsilon > 0 \), existe una partición \( P_1 \) tal que:  
		\[
		U(f, P_1) < \overline{I} + \frac{\varepsilon}{2} = I + \frac{\varepsilon}{2}.
		\]  
		- Existe una partición \( P_2 \) tal que:  
		\[
		L(f, P_2) > \underline{I} - \frac{\varepsilon}{2} = I - \frac{\varepsilon}{2}.
		\]  
		
		Sea \( P_\varepsilon = P_1 \cup P_2 \) (afinamiento común de \( P_1 \) y \( P_2 \)). Por propiedades de las sumas de Darboux:  
		\[
		U(f, P_\varepsilon) \leq U(f, P_1) < I + \frac{\varepsilon}{2},
		\]  
		\[
		L(f, P_\varepsilon) \geq L(f, P_2) > I - \frac{\varepsilon}{2}.
		\]  
		Por lo tanto:  
		\[
		U(f, P_\varepsilon) - L(f, P_\varepsilon) < \left(I + \frac{\varepsilon}{2}\right) - \left(I - \frac{\varepsilon}{2}\right) = \varepsilon.
		\]  
		Sea \( P' \supseteq P_\varepsilon \) un afinamiento de \( P_\varepsilon \). Para cualquier elección de puntos \( \chi \):  
		\[
		L(f, P_\varepsilon) \leq L(f, P') \leq \sigma(f, P', \chi) \leq U(f, P') \leq U(f, P_\varepsilon).
		\]  
		Dado que \( I = \overline{I} = \underline{I} \), tenemos:  
		\[
		I - \frac{\varepsilon}{2} < L(f, P_\varepsilon) \leq L(f, P') \leq \sigma(f, P', \chi) \leq U(f, P') \leq U(f, P_\varepsilon) < I + \frac{\varepsilon}{2}.
		\]  
		Esto implica:  
		\[
		|\sigma(f, P', \chi) - I| < \frac{\varepsilon}{2} < \varepsilon.
		\]  
		Conclusión
		Si \( \overline{I} = \underline{I} \), entonces para todo \( \varepsilon > 0 \), existe una partición \( P_\varepsilon \) tal que cualquier refinamiento \( P' \supseteq P_\varepsilon \) satisface:  
		\[
		|\sigma(f, P', \chi) - I| < \varepsilon.
		\]  
		Por lo tanto, \( f \) es Riemann integrable en \([a, b]\).  
		
		\[
		\boxed{\overline{I} = \underline{I} \implies f \in \mathcal{R}[a, b]}
		\]
		
		
	\end{proof}
	
	Por Tanto:\\
		\boxed{ f \in R:[a,b] \iff \forall \varepsilon >0   [\exists P_\varepsilon :  P' \supseteq P_\varepsilon 	|\sigma(f, P', \chi) - I| < \varepsilon] \iff 	\overline{ I } = \underline{I}.}
	
	
\end{document}
